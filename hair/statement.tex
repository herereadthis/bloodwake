% \documentclass[14pt, letterpaper]{article}
\documentclass[12pt, letterpaper]{extarticle}

\usepackage[letterpaper, inner=1.5in, outer=1.5in, top=1.25in, bottom=1in]{geometry}

\usepackage{fouriernc}

\usepackage[english]{babel}
\usepackage[utf8]{inputenc}
% \usepackage{blindtext}

% make text look better
\usepackage{microtype}

% prevents hyphenation
\usepackage[none]{hyphenat} 

% use images
\usepackage{graphicx}

% wrap text around images
% \usepackage{wrapfig}

% make lists look better
% \usepackage{enumitem}

% fancy headers
% \usepackage{fancyhdr}

% math formulas
% \usepackage{amsmath}

% auto-generate indices
% usepackage{index}

% typography
% ragged right, left aligned
\usepackage[document]{ragged2e}
% control paragaph indents
\setlength{\parindent}{0em}
% line spacing
\renewcommand{\baselinestretch}{1.0}

% remove page numbering
\pagenumbering{gobble}

 \begin{document}
 \title{\LARGE{The Hair Project}}
 \author{\normalsize{April 2023}}
 \date{}
  
\maketitle

\vspace{1\baselineskip}
We, the students of Photography Class 201+202, have collaborated on a hairy project.

\vspace{1\baselineskip}
\textit{Hair is everywhere.} Is very nearby. You do not have to go looking for it. You can find it at home. You don't have to go buy it (unless you want to). You don't have to wait for a certain time of year for it to appear. You don't have to travel far to find it.

\vspace{1\baselineskip}
\textit{Hair is personality.} You can tell stories about a person by looking at their hair. What if there is missing hair, or fading hair? You can cut it the length you want. If you want a new color, you can make hair a new color. If you don't like it straight, then you can make it not straight. If you don't like it curly, you can make it not curly.

\vspace{1\baselineskip}
We, the students of Photography Class 201+202, have put together a hairy show: the curation, the presentation, the gallery, everything. We did it in 2 months.

\vspace{1\baselineskip}
\textit{Hair is abstract.} If it's not on your head, it be tangled into a brush, or caught on the side of a sink. With so much hair, can we instead look at the patterns in hair, or the texture of hair? Can a picture of hair just be about hair and not about the person from which the hair came?

\vspace{1\baselineskip}
\textit{Hair is identity.} What is known about a person's heritage or ancestry from their hair? Does the identity of our hair affect our social interactions, our work, or the friends we choose? Should we choose to, then hopefully we feel empowered to express our identity through our hair. Or perhaps sometimes, we feel pressured to hide or disguise our hair's identity.





% \vspace{1\baselineskip}
% 13 December, 2021
  
  \end{document}
  